\section{Analisi}

\subsection{Utenza finale}
Il sito, sulla base del suo dominio applicativo, dovrebbe essere frequentato perlopiù da utenti la cui età
si aggira tra i 20 e i 60 anni. Si può quindi affermare
che la quasi totalità degli utenti, se non tutta, ha familiarità con la navigazione sul web e più in
generale con l'utilizzo dei dispositivi adatti allo scopo, primo fra tutti lo smartphone, che risulta
essere il mezzo più comunemente utilizzato per navigare su Internet.\\
\\
Si presume inoltre che gli utenti studino o lavorino in città, o che risiedano a Padova o in provincia. Gli 
utenti dovrebbero conoscere quindi, quantomeno in modo approssimativo, la vicinanza o meno dal locale.\\
\\
Non ci sono altri fattori per classificare l'utenza del sito; infatti sotto altri punti di vista (come
genere, lavoro, passioni, ecc...) l'utenza è variegata. Per questo motivo il linguaggio utilizzato
all'interno del sito non deve essere tecnico ma semplice e alla portata di tutti.

\subsection{Motori di ricerca}
Per quanto riguarda gli utenti che conoscono bene il locale, è plausibile pensare che cerchino il
locale sui motori di ricerca per nome. Pertanto la query a cui sicuramente il sito deve rispondere è
il nome stesso del ristorante, come \textit{Pizzeria Ai 4 Laureandi}, \textit{Ai 4 Laureandi Padova},
\textit{Pizzeria Ai 4 Laureandi Padova}.\\
\\
Tuttavia è corretto sostenere che non tutti gli utenti del sito siano assidui frequentatori del posto e lo
conoscano; potrebbe anzi essere che l'utente approdi nel sito dopo aver effettuato una ricerca su Internet
per scegliere un ristorante in cui cenare ad esempio. Ê quindi importante che il sito abbia
un buon posizionamento sui motori di ricerca per battere la concorrenza. Sicuramente il sito dovrebbe
rispondere a tutte le query che contengono le parole \textit{pizzeria, Padova, pizzeria Padova}.