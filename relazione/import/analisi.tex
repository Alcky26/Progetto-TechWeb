\section{Analisi}

\subsection{Utenza finale}
Il sito, sulla base del suo dominio applicativo, dovrebbe essere frequentato perlopiù da utenti la cui età
si aggira tra i 20 e i 60 anni. Si può quindi affermare
che la quasi totalità degli utenti, se non tutta, ha familiarità con la navigazione sul web e più in
generale con l'utilizzo dei dispositivi adatti allo scopo, primo fra tutti lo smartphone, che risulta
essere il mezzo più comunemente utilizzato per navigare su Internet.\\
\\
Si presume inoltre che gli utenti studino o lavorino in città, o che risiedano a Padova o in provincia. Gli 
utenti dovrebbero conoscere quindi, quantomeno in modo approssimativo, la vicinanza o meno dal locale.\\
\\
Non ci sono altri fattori per classificare l'utenza del sito; infatti sotto altri punti di vista (come
genere, lavoro, passioni, ecc...) l'utenza è variegata. Per questo motivo il linguaggio utilizzato
all'interno del sito non deve essere tecnico ma semplice e alla portata di tutti.

\subsection{Motori di ricerca}
Per quanto riguarda gli utenti che conoscono bene il locale, è plausibile pensare che cerchino il
locale sui motori di ricerca per nome. Pertanto la query a cui sicuramente il sito deve rispondere è
il nome stesso del ristorante, come \textit{Pizzeria Ai 4 Laureandi}, \textit{Ai 4 Laureandi Padova},
\textit{Pizzeria Ai 4 Laureandi Padova}.\\
\\
Tuttavia è corretto sostenere che non tutti gli utenti del sito siano assidui frequentatori del posto e lo
conoscano; potrebbe anzi essere che l'utente approdi nel sito dopo aver effettuato una ricerca su Internet
per scegliere un ristorante in cui cenare ad esempio. Ê quindi importante che il sito abbia
un buon posizionamento sui motori di ricerca per battere la concorrenza. Sicuramente il sito dovrebbe
rispondere a tutte le query che contengono le parole \textit{pizzeria, Padova, pizzeria Padova}.

\subsection{Casi d'uso}
\subsubsection{Utente generico}
Si definisce generico un utente che:
\begin{itemize}
	\item Visita il sito senza possedere un account;
	\item Visita il sito senza effettuare il login.
\end{itemize}
I casi d'uso relativi all'utente generico sono:
\begin{itemize}
	\item \textit{Visualizzazione homepage del sito (UC1)}: l'utente accede alla pagina principale
	recandosi alla pagina apposita collegandosi al sito, cliccando il logo del locale posto nell'header,
	oppure cliccando il link corrispondente presente nel menù di navigazione secondario posto nel footer;
	\item \textit{Visualizzazione informazioni sul locale (UC2)}: l'utente consulta le informazioni circa
	il locale recandosi alla pagina apposita cliccando sul link corrispondente presente nell'header o in
	alternativa nel menù di navigazione secondario posto nel footer;
	\item \textit{Visualizzazione elementi del listino (UC3)}: l'utente visualizza gli elementi del
	listino che è possibile consumare o acquistare presso il locale, recandosi alla pagina apposita
	cliccando sul link corrispondente presente nell'header o in alternativa nel menù di navigazione
	secondario posto nel footer;
	\begin{itemize}
		\item \textit{Visualizzazione singolo elemento (UC3.1)}: l'utente visualizza un singolo elemento
		del listino;
		\begin{itemize}
			\item \textit{Visualizzazione nome singolo elemento (UC3.1.1)}: l'utente visualizza il nome
			di un singolo elemento del listino;
		\end{itemize}
		\item \textit{Visualizzazione pizza (UC3.2)}: l'utente visualizza le informazioni inerenti ad
		una pizza;
		\begin{itemize}
			\item \textit{Visualizzazione ingredienti pizza (UC3.2.1)}: l'utente visualizza gli
			ingredienti di cui è composta una pizza, e per ognuno la categoria di allergene a cui
			appartiene;
		\end{itemize}
		\item \textit{Visualizzazione pizza classica (UC3.3)}: l'utente visualizza le informazioni
		inerenti ad una pizza classica;
		\item \textit{Visualizzazione pizza speciale (UC3.4)}: l'utente visualizza le informazioni
		inerenti ad una pizza speciale;
		\item \textit{Visualizzazione calzone (UC3.5)}: l'utente visualizza le informazioni
		inerenti ad un calzone;
		\item \textit{Visualizzazione bevanda (UC3.6)}: l'utente visualizza le informazioni inerenti ad
		una bevanda;
		\item \textit{Visualizzazione bevanda analcolica (UC3.7)}: l'utente visualizza le informazioni
		inerenti ad una bevanda analcolica;
		\item \textit{Visualizzazione bevanda alcolica (UC3.8)}: l'utente visualizza le informazioni
		inerenti ad una bevanda alcolica;
		\begin{itemize}
			\item \textit{Visualizzazione grado alcolico bevanda (UC3.8.1)}: l'utente visualizza il
			grado alcolico della bevanda;
			\item \textit{Visualizzazione descrizione bevanda (UC3.8.2)}: l'utente visualizza una breve
			descrizione della bevanda in termini di gusto e sapori;
		\end{itemize}
		\item \textit{Visualizzazione birra (UC3.9)}: l'utente visualizza le informazioni inerenti ad
		una birra;
		\item \textit{Visualizzazione vino (UC3.10)}: l'utente visualizza le informazioni inerenti ad
		un vino;
		\item \textit{Visualizzazione dolce (UC3.11)}: l'utente visualizza le informazioni inerenti ad
		un dolce;
	\end{itemize}
	\item \textit{Visualizzazione orari locale (UC4)}: l'utente consulta i giorni e gli orari in
	cui il locale è aperto recandosi alla pagina apposita cliccando sul link corrispondente presente
	nell'header o in alternativa nel menù di navigazione secondario posto nel footer;
	\item \textit{Visualizzazione ubicazione locale (UC5)}: l'utente visualizza su una mappa dove si
	trova fisicamente la pizzeria recandosi alla pagina apposita cliccando sul link corrispondente
	presente nell'header o in alternativa
	nel menù di navigazione secondario posto nel footer;
	\item \textit{Registrazione (UC6)}: l'utente registra un proprio account nel sito recandosi alla
	pagina apposita cliccando sul link corrispondente presente nell'header o in alternativa nel menù di
	navigazione secondario posto nel footer e successivamente inserendo credenziali valide;
	\item \textit{Login (UC7)}: l'utente effettua l'accesso con il proprio account nel sito recandosi
	alla pagina apposita cliccando sul link corrispondente presente nell'header, o venendoci
	reindirizzato nel caso si voglia usufruire di un servizio che richiede l'aver effettuato l'accesso, o
	in alternativa cliccando sul link presente nel menù di
	navigazione secondario posto nel footer. Successivamente l'utente inserisce le sue credenziali. Nel
	caso in cui il login va a buon fine l'utente viene indirizzato alla pagina "area utente". Nel
	caso in cui il login fallisce l'utente viene reindirizzato alla pagina di login e verrà mostrato un
	messaggio di errore del tipo "Credenziali non valide".
\end{itemize}
\subsubsection{Utente loggato}
Un utente loggato è colui che ha effettuato correttamente l'accesso nel sito. Oltre ad ereditare i casi
d'uso dell'utente generico tranne \textit{(UC6)} ed \textit{(UC7)}, dispone anche dei seguenti:
\begin{itemize}
	\item \textit{Ordinazione d'asporto (UC8)}: l'utente ordina d'asporto recandosi alla pagina apposita
	cliccando sul link corrispondente presente nell'header o in alternativa nel menù di navigazione
	secondario posto nel footer;
	\begin{itemize}
		\item \textit{Scelta quantità elementi listino (UC8.1)}: l'utente per ogni
		elemento del listino sceglie la quantità che desidera acquistarne;
		\begin{itemize}
			\item \textit{Scelta quantità singolo elemento (UC8.1.1)}: l'utente sceglie la quantità che
			desidera acquistare di un elemento del listino;
			\item \textit{Visualizzazione importo totale (UC8.1.2)}: l'utente visualizza l'importo totale
			di quello che intende ordinare;
		\end{itemize}
		\item \textit{Scelta modalità di ritiro (UC8.2)}: l'utente sceglie se ritirare in loco o se
		utilizzare il servizio di delivery;
		\begin{itemize}
			\item \textit{Scelta orario di ritiro (UC8.3)}: l'utente sceglie l'ora in cui ritirare
			l'ordine;
		\end{itemize}
		\item \textit{Scelta ritiro in loco (UC8.3)}: l'utente sceglie di ritirare l'ordine presso il
		locale;
		\item \textit{Scelta delivery (UC8.4)}: l'utente sceglie di usufruire del servizio di delivery;
		\begin{itemize}
			\item \textit{Scelta ubicazione ritiro (UC8.4.1)}: l'utente fornendo un indirizzo di
			domicilio valido sceglie dove ricevere l'ordine effettuato;
		\end{itemize}
		\item \textit{Pagamento (UC8.5)}: l'utente paga la somma dovuta;
		\item \textit{Pagamento tramite carta di credito (UC8.6)}: l'utente paga la somma dovuta
		inserendo i dati relativi ad una carta di credito valida;
		\item \textit{Pagamento tramite bonus (UC8.7)}: l'utente paga la somma dovuta spendendo un bonus
		il cui valore sia sufficiente a coprire l'acquisto;
	\end{itemize}
	\item \textit{Prenotazione tavolo (UC9)}: l'utente prenota un tavolo presso la pizzeria recandosi
	alla pagina apposita cliccando sul link corrispondente presente
	nell'header o in alternativa nel menù di navigazione secondario posto nel footer;
	\begin{itemize}
		\item \textit{Scelta giorno ed ora (UC9.1)}: l'utente sceglie giorno ed ora per cui intende
		prenotare un tavolo;
		\item \textit{Scelta numero persone (UC9.2)}: l'utente sceglie quante persone occuperanno
		il tavolo;
	\end{itemize}
	\item \textit{Visualizzazione informazioni personali (UC10)}: l'utente visualizza le sue informazioni
	recandosi alla pagina apposita cliccando sul link corrispondente presente
	nell'header o in alternativa nel menù di navigazione secondario posto nel footer;
	\item \textit{Modifica informazioni personali (UC11)}: l'utente modifica le sue informazioni
	recandosi alla pagina apposita cliccando sul link corrispondente presente
	nell'header o in alternativa nel menù di navigazione secondario posto nel footer.
\end{itemize}
\subsubsection{Utente amministratore}
Un utente amministratore è colui che ha effettuato correttamente l'accesso al sito e possiede i permessi
di amministratore. Oltre ad ereditare i casi d'uso di utente generico ed utente loggato tranne
\textit{(UC6)} ed \textit{(UC7)}, dispone anche dei seguenti:
\begin{itemize}
	\item \textit{Aggiunta elementi al listino (UC12)}: l'utente aggiunge nuovi elementi al listino
	recandosi alla pagina apposita cliccando sul link corrispondente presente
	nell'header o in alternativa nel menù di navigazione secondario posto nel footer;
	\item \textit{Rimozione elementi dal listino (UC13)}: l'utente rimuove elementi dal listino
	recandosi alla pagina apposita cliccando sul link corrispondente presente
	nell'header o in alternativa nel menù di navigazione secondario posto nel footer;
	\item \textit{Modifica elementi del listino (UC14)}: l'utente modifica le informazioni relative agli
	elementi del listino recandosi alla pagina apposita cliccando sul link corrispondente presente
	nell'header o in alternativa nel menù di navigazione secondario posto nel footer;
	\item \textit{Modifica disponibilità elementi del listino (UC15)}: l'utente rende non disponibili
	degli elementi del listino in modo che non compaiano nell'elenco, recandosi alla pagina apposita
	cliccando sul link corrispondente presente nell'header o in alternativa nel menù di navigazione
	secondario posto nel footer;
	\item \textit{Aggiunta ingredienti (16)}: l'utente aggiunge nuovi ingredienti recandosi alla pagina
	apposita cliccando sul link corrispondente presente nell'header o in alternativa nel menù di
	navigazione secondario posto nel footer;
	\item \textit{Rimozione ingredienti (17)}: l'utente rimuove ingredienti recandosi alla pagina
	apposita cliccando sul link corrispondente presente nell'header o in alternativa nel menù di
	navigazione secondario posto nel footer;
	\item \textit{Modifica ingredienti (18)}: l'utente modifica le informazioni relative ad ingredienti
	recandosi alla pagina apposita cliccando sul link corrispondente presente nell'header o in
	alternativa nel menù di navigazione secondario posto nel footer;
\end{itemize}