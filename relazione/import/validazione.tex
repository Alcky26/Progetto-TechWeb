\section{Validazione}

\subsection{W3C HTML Validator}
Lo strumento di validazione di \textit{W3C} permette di validare il codice \textit{HTML}, anche quello
prodotto da script \textit{PHP} visto che è possibile validare semplicemente facendo un copia-incolla del
codice. Tuttavia non è stato possibile validare tutto il codice \textit{HTML} prodotto da \textit{PHP}
viste le numerose combinazioni di input.

\subsection{W3C CSS Validator}
Questo strumento è stato utilizzato per validare il codice \textit{CSS}, il quale deve rispettare
rigorosamente lo standard.

\subsection{WAVE Evaluation Tool}
Questo strumento è stato preso in considerazione fin da subito, infatti ha permesso di scovare problemi
legati all'accessibilità. Inoltre il suo utilizzo è molto semplice, perché oltre al sito
il servizio può essere fruibile anche tramite estensioni installabili sui browser più comunemente
utilizzati (es. Mozilla Firefox).

\subsection{PhpCodeChecker}
Grazie all'utilizzo di questo tool è stato possibile verificare la sintassi di tutti i file \textit{PHP}.

\subsection{Esprima}
Questo strumento è servito a controllare la sintassi degli script JavaScript.

\subsection{contrast-ratio.com}
Il servizio offerto da questo sito permette di determinare il livello di contrasto tra due colori, di cui
uno di background e l'altro del font, e ha consentito al gruppo di valutare la bontà della scelta della
combinazione di colori.