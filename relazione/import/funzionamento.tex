\section{Funzionamento}

La parte pubblica del sito comprende, oltre alla pagina principale, la pagina di presentazione del locale
e quella dei contatti, e sono visitabili da utenti loggati e non. Anche le pagine di menù e di prenotazione
di un tavolo sono pubbliche, però l'usufruire dei servizi di ordinazione online e di
prenotazione effettiva richiede come precondizione l'aver effettuato accesso.\\
\\
Nel caso in cui l'utente non possegga un account, può crearne uno attraverso il form di registrazione.
Una volta creato è possibile effettuare l'accesso attraverso le proprie credenziali e poter quindi:
\begin{itemize}
	\item Accedere alla propria area riservata per:
	\begin{itemize}
		\item Visualizzare e modificare le proprie credenziali, così come le proprie informazioni
		personali;
		\item Visualizzare e in aggiunta filtrare le proprie prenotazioni di tavoli;
		\item Visualizzare e in aggiunta filtrare i propri acquisti;
		\item Visualizzare e in aggiunta filtrare i propri bonus.
	\end{itemize}
	\item Ordinare online (il pagamento chiaramente viene simulato, tranne nel caso di pagamento con
	bonus, in quel caso il bonus verrà segnato come speso);
	\item Prenotare un tavolo.
\end{itemize}
L'utente amministratore, oltre ai permessi sopra citati, può:
\begin{itemize}
	\item Aggiungere, rimuovere o modificare elementi del listino;
	\item Modificare la composizione degli ingredienti di una pizza;
\end{itemize}